%!TEX root=./LIVRO.tex

\chapter{Introdução}

%\chapter{Os inimigos do homem serão as pessoas de sua própria casa: \emph{crítica e apologia sociais em ``Pai contra mãe'', de Machado de
%Assis}}

\begin{flushright}
\emph{Alexandre Rosa}\footnote{Alexandre Rosa, escritor e
  cientista social formado pela \versal{FFLCH-USP}, é mestre em Literatura
  Brasileira pelo Instituto de Estudos Brasileiros da \versal{USP}. Participou da
  Coleção Ensaios Brasileiros Contemporâneos -- Volume Música (Funarte,
  2017), com o ensaio \emph{Três Raps de São Paulo}, em parceria com
  Guilherme Botelho e Walter Garcia.}

\emph{Flávio Ricardo Vassoler}\footnote{Flávio Ricardo Vassoler
  é doutor em Letras pela \versal{FFLCH-USP}, com estágio doutoral junto à
  Northwestern University (\versal{EUA}). É autor das obras literárias \emph{Tiro
  de Misericórdia} (nVersos, 2014) e \emph{O Evangelho segundo Talião}
  (nVersos, 2013) e organizador do livro de ensaios \emph{Dostoiévski e
  Bergman: o niilismo da modernidade} (Intermeios, 2012).}

\emph{Ieda Lebensztayn}\footnote{Ieda Lebensztayn é
  pesquisadora de pós-doutorado na Biblioteca Brasiliana Mindlin,
  \versal{BBM-USP/ FFLCH-USP} (Processo \versal{CNP}q 166032/2015-8), com estudo a
  respeito da recepção literária de Machado de Assis. Doutora em
  Literatura Brasileira pela \versal{FFLCH-USP}, fez pós-doutorado no \versal{IEB-USP}
  sobre a correspondência de Graciliano Ramos. Autora de
  \emph{Graciliano Ramos e a} Novidade\emph{: o astrônomo do inferno e
  os meninos impossíveis} (Hedra, 2010). Organizou, com Thiago Mio
  Salla, os livros \emph{Cangaços} e \emph{Conversas}, de Graciliano
  Ramos, publicados em 2014 pela Record.}
\end{flushright}

\begin{quote}
Não julgueis que vim trazer a paz à terra. Vim trazer não a paz, mas a
espada. Eu vim trazer a divisão entre o filho e o pai, entre a filha e a
mãe, entre a nora e a sogra, e os inimigos do homem serão as pessoas de
sua própria casa\footnote{``Evangelho segundo Mateus'', capítulo 10,
  versículos 34 a 36. In: \emph{Bíblia sagrada.} Tradução dos originais
  mediante a versão dos monges de Maredsous (Bélgica) pelo Centro
  Bíblico Católico. São Paulo: Editora Ave-Maria, 1994, p. 1296.}.
\end{quote}

\section{Preâmbulo}

No princípio era o verbo? Não. No princípio eram o choro e o ranger de
dentes.

Como um obstetra que nos dá as boas-vindas a este mundo com um tapa que
nos faz chorar -- \emph{bem-vindos ao nosso vale de lágrimas} --, o
narrador machadiano de ``Pai contra mãe'' rasga o ventre de seu conto
sentenciando que ``a ordem social e humana nem sempre se alcança sem o
grotesco, e alguma vez o cruel''.

Estaríamos, então, inequivocamente envoltos por um \emph{ethos} -- ou
melhor, um \emph{pathos} -- machadiano a referendar a (e a se resignar
diante da) iniquidade da história e da natureza humanas. Ora, mas e se a
dubiedade do olhar de cigana oblíqua e dissimulada de Capitu --
dubiedade com a qual só conseguimos entrar em contato por meio do relato
deveras enviesado de Bentinho, o \emph{Dom Casmurro} potencialmente
traído por Capitu -- pudesse se confundir com a própria \emph{estrutura
narrativa} de ``Pai contra mãe''? Nesse caso, apologia e crítica sociais
estariam umbilical e incestuosamente enredadas, o que nos permitiria
insinuar que o narrador machadiano, para quem o grotesco e o cruel
seriam as balizas (o arame farpado) de nossa história, seria
\emph{dialeticamente pessimista:} no ventre do conto a arremessar o pai
contra a mãe, haveria também a possibilidade de revelar (e criticar) a
ordem social que pressupõe o grotesco e o cruel como sentinelas
inequívocas.

Analisemos, então, como o joio da apologia e o trigo da crítica sociais
se fundem e se confundem em ``Pai contra mãe''. Apenas não percamos de
vista que a tentativa de \emph{separar} o joio do trigo, isto é, o
ímpeto de arregimentar Machado de Assis, inequivocamente, \emph{ou} como
menestrel do niilismo, \emph{ou} como estandarte da revolução tende a
empobrecer o caráter polissêmico de sua estrutura narrativa a acompanhar
as \emph{contradições} de uma ordem social e humana que, até hoje, ainda
não conseguiu prescindir do grotesco e do cruel como sentinelas de nossa
história.

* * * * * *

Em meados do século \versal{XIX}, na cidade do Rio de Janeiro, o narrador
cinicamente crítico -- e/ou criticamente cínico -- de ``Pai contra mãe''
nos diz que ``os escravos fugiam com frequência. Eram muitos, e nem
todos gostavam da escravidão. Sucedia ocasionalmente apanharem pancada,
e nem todos gostavam de apanhar pancada''. Ora, se os escravos fugiam
com frequência, muitos repudiavam os aguilhões da escravidão -- mas,
ainda assim, como \emph{nem todos} gostavam da escravidão e como
\emph{nem todos} gostavam de apanhar pancada, podemos deduzir que havia
também, sempre segundo as (contra)informações do narrador machadiano,
aqueles que rezavam segundo o discurso da servidão voluntária. O curioso
(e sintomático) é que grande parte dos escravos fujões era apenas
repreendida, já que poderia haver ``alguém de casa que servia de
padrinho {[}ao escravo{]}'', e talvez o dono não fosse ``mau; além
disso, o sentimento de propriedade modera a ação, porque dinheiro também
dói''. Assim, a moderação da ira senhorial -- moderação que, por vezes,
lançava mão do chicote e do pelourinho como instrumentos de catequese --
ocorria não pela mediação da \emph{Declaração dos Direitos do Homem e do
Cidadão}, cujo conteúdo emancipatório ainda não chegara ao Brasil
escravocrata de então, mas pelo prejuízo pecuniário que o senhor poderia
sofrer ao avariar sua mercadoria escrava. Ora, nós bem poderíamos ler
essa verdadeira chicotada -- ``dinheiro também dói'' -- como uma crítica
sumamente epigráfica à ordem social que administrava seres humanos como
coisas. Mas, como em Machado de Assis a ordem social é (retro)alimentada
pelo legado de nossa miséria ontológica, o capitalismo à brasileira
também poderia ser a luva a calçar a mão da natureza humana. Como tal
ferida histórica não apenas não foi cicatrizada como parece expelir cada
vez mais pus -- a dubiedade machadiana bem poderia insinuar que as
tentativas reformistas e/ou revolucionárias de amenizar nossas feridas
históricas com mertiolate também visaram provocar ainda mais agonia e
ardência no corpo social --, a atualidade da \emph{crítica social
apologista} de ``Pai contra mãe'' parece residir em sua capacidade de
escarafunchar contradições ainda não superadas -- para um sem-número de
personagens machadianas, trata-se de \emph{contradições}
\emph{insuperáveis}.

Paridos a fórceps o grotesco e o cruel do conto machadiano que
arremessará o pai contra a mãe, ficamos sabendo que, em meio à sociedade
brasileira oitocentista encimada por uma exígua cúpula de senhores e
assentada sobre o dorso prostrado da escravidão, ``pegar escravos
fugidios era um ofício do tempo''. Mas, ora, quem eram os atores sociais
que se aventuravam por tais veredas esguias e o que os levava a um
ofício tão incerto? Nosso narrador prontamente nos revela que

\begin{quote}
ninguém se metia em tal ofício por desfastio ou estudo; a pobreza, a
necessidade de uma achega, a inaptidão para outros trabalhos, o acaso e,
alguma vez, o gosto de servir também, ainda que por outra via, davam o
impulso ao homem que se sentia bastante rijo para pôr ordem à desordem.
\end{quote}

É assim que, para Cândido Neves, descendente do \emph{ethos} macunaímico
de Leonardo Pataca, o anti-herói -- ou, por outra, o \emph{herói à
brasileira --} de \emph{Memórias de um sargento de milícias} (1854), de
Manuel Antônio de Almeida, o ofício -- ou melhor, o \emph{biscate} -- de
captor eventual de escravos era como a ocasião que faz o ladrão, já que
Candinho era acometido pela síndrome do \emph{caiporismo}, isto é, o
rapaz não parava quieto nos (sub)empregos que, vez por outra, ele
amealhava entre um fiado e outro no boteco, entre um jantar e outro na
casa de parentes e amigos. Assim, como os donos dos escravos fugidos
prometiam gratificações generosas para os captores em seus anúncios nos
jornais, o caiporismo e o dinheiro fácil faziam com que Candinho
ressignificasse o anátema divino do Velho Testamento: \emph{Ganharás o
pão sem o suor do teu rosto. }

Não deixemos de notar que, em meio ao darwinismo social à brasileira, a
lei dos senhores brancos só fazia legar o salve-se quem puder à legião
de espoliados pardos, mestiços e negros. Assim, a crueldade machadiana
faz com que o (anti-~)herói de ``Pai contra mãe'' seja submetido à dupla
eugenia de se chamar \emph{Cândido Neves.} E, como se tal tentativa de
arrefecer/embranquecer a negritude como destino social não fosse
suficiente, o grotesco machadiano faz com que Candinho se apaixone pela
costureira \emph{Clara.} Em face do casamento do caipora com a
costureira, Mônica, tia de Clara, não tem muita dificuldade em fazer as
vezes de pitonisa: ``Vocês, se tiverem um filho, morrem de fome''. Não
deixemos de notar, ademais, que,

\begin{quote}
desde a Idade Média, [o nome Mônica] tem sido associado ao termo
latino \emph{moneo}, que quer dizer \emph{conselheiro}, e ao termo grego
\emph{monos}, que significa \emph{um, único}. No século \versal{IV} d.C., esse
nome surge a partir da santa norte-africana Mônica de Hipo, mãe de Santo
Agostinho, a quem ela converteu ao cristianismo\footnote{\emph{Behind
  the Name: The Etymology and History of First Names} [Por detrás dos
  nomes: a etimologia e a história dos nomes]. \emph{Mônica: https://www.behindthename.com/name/mo13nica}.
  Consulta feita no dia 21/10/17.}.
\end{quote}

Como a crueldade machadiana e o grotesco da realidade social parecem não
ter fim, a Mônica do nosso conto é a \emph{conselheira singular} que
traz a divisão entre o pai e a mãe -- na paródia de Machado de Assis, o
nome da mãe de Agostinho de Hipona, canonizado pelas \emph{Confissões}
(397-398 d.C.) do teólogo católico, invoca o anátema de que o filho de
Cândido e Clara seja conduzido não à pia batismal, mas à Roda dos
Enjeitados.

No princípio era o verbo -- abortar.

Diante da crescente concorrência com a legião de caiporas/captores de
escravos e costureiras que faz minguar os ganhos já exíguos de Candinho
e Clara; diante da ordem de despejo iminente do locatário do casebre;
diante da comida cada vez mais irregular e escassa, a tia e apóstata
Mônica prega o 11º Mandamento: \emph{Abortarás. }

Eis, então, o que o narrador machadiano nos revela em um trecho que
parece extraído do \emph{Evangelho segundo Mônica Iscariotes: }

\begin{quote}
A situação era aguda. {[}Candinho e Clara{]} Não achavam casa nem
contavam com pessoa que lhes emprestasse alguma; era ir para a rua. Não
contavam com a tia. Tia Mônica teve a arte de alcançar aposento para os
três em casa de uma senhora velha e rica, que lhe prometeu emprestar os
quartos baixos da casa, ao fundo da cocheira, para os lados de um pátio.
Teve ainda a arte maior de não dizer nada aos dois, para que Cândido
Neves, no desespero da crise, começasse por enjeitar o filho e acabasse
alcançando algum meio seguro e regular de obter dinheiro; emendar a
vida, em suma. Ouvia as queixas de Clara, sem as repetir, é certo, mas
sem as consolar. No dia em que fossem obrigados a deixar a casa,
fá-los-ia espantar com a notícia do obséquio e iriam dormir melhor do
que cuidassem.
\end{quote}

Para abrigar Candinho e Clara, tia Mônica Iscariotes espera que o casal
seja despejado; para que Candinho abandone de vez o caiporismo, tia
Mônica Iscariotes espera que o pai enjeite o filho; ainda que tenha uma
carta na manga para impedir que Candinho e Clara durmam ao relento, tia
Mônica Iscariotes ouve as queixas de Clara, mas não consola a sobrinha.
Se, para a tia Mônica dos Anjos, \emph{a mão que afaga é a mesma que
apedreja}; se, para a tia Mônica Citotec, é preciso jogar o bebê fora
junto com a água do banho para salvaguardar a banheira; se, para a tia
Mônica Maquiavel, os fins legitimam os meios, bem podemos entrever por
que, no \emph{Evangelho segundo Mônica Iscariotes}, Judas trai Jesus
Cristo com um beijo.

Ocorre que, mesmo com todo o caiporismo de Candinho, o rapaz
\emph{parece} ter verdadeira afeição por Clara e, antes mesmo do
nascimento do bebê, Candinho já \emph{parece} amá-lo enternecidamente.
Neste momento, os leitores escolados de Machado de Assis já tentam
encontrar alguma fissura no amor paterno de Candinho por conta da
utilização do verbo \emph{parecer.} Na verdade, logo veremos que o amor
de pai -- amor que não deixa de ter seus laivos narcísicos -- se
afirmará, grotescamente, contra o amor de mãe, já que ``nem todas as
crianças vingam''. Mas, antes de chegarmos à vitória de Pirro --
\emph{ao vencedor, as batatas} -- do filho do captor de escravos
Candinho sobre o filho da escrava Arminda; antes de chegarmos, a bem
dizer, a um dos desenlaces mais cruéis da obra de Machado de Assis,
voltemos ao calvário dos pais Cândido Neves e Clara, já que o filho do
casal acaba de nascer. ``Notai que era um menino e que ambos os pais
desejavam justamente este sexo''. \emph{Escarra na boca que te beija:} o
narrador machadiano faz com que tal afeição tenha a dupla (e dúbia)
função de enternecer o coração dos pais ao mesmo tempo em que torna
ainda mais árdua a \emph{via crucis} do filho rumo à Roda dos
Enjeitados. Mas, ora, diante da penúria, que fazer? Para que consigamos
oferecer a outra face, é preciso que todos e cada um de nós tenhamos um
rosto -- o espectro da tia Mônica Iscariotes só faz sentenciar que o
velho dito de que \emph{onde comem dois também comem três} não compra a
fiado nem na mercearia do irmão do padre. Então, diante da penúria, que
fazer?

Fazer, executar. 11º Mandamento: \emph{Abortarás. }

Assim, Clara e Candinho mal puderam dar algum leite ao bebê -- chegara a
hora de enjeitá-lo de vez. Mas, ``como chovesse à noite, assentou o pai
levá-lo à Roda na noite seguinte''.

Repleto de comiseração pelo filho, Candinho aproveita a noite derradeira
para ler e reler, um a um, os últimos anúncios que prometiam recompensas
para captores de escravos fugidos. A maioria lhe pareceu fogo fátuo --
meras promessas ou gratificações escassas. Uma recompensa por uma
mulata, no entanto, subia à polpuda soma de cem mil-réis. Na manhã
seguinte, Candinho se embrenhou pelo centro do Rio de Janeiro à caça de
pistas da escrava fugida, mas nada logrou descobrir. Quando do triste
retorno ao casebre em que morava de favor após o despejo, Candinho
encontra tia Mônica com o bebê já pronto para ser levado à Roda dos
Enjeitados. Diante da miséria patente e da resignação cabisbaixa de
Clara, o pai decide abortar o filho.

Reparemos que a Roda dos Enjeitados ficava ``na direção da rua dos
Barbonos'', cujo nome alude aos ``membros da Ordem dos Frades Menores
Capuchinhos, {[}uma{]} ordem religiosa franciscana e
reformada''\footnote{Aurélio Buarque de Holanda Ferreira, \emph{Aurélio:
  o dicionário da língua portuguesa.} Curitiba: Editora Positivo, 2004,
  p. 167.}. Como uma faca só lâmina -- a mesma faca que degolará o filho
de Candinho --, a crueldade machadiana e o grotesco da ordem social e
humana só fazem abortar a bondade e a compaixão que remontam a São
Francisco de Assis.

Candinho fizera com que Clara amamentasse o filho uma última vez antes
de enjeitá-lo; o pai queria levar o filho de volta para casa enquanto
percorria a \emph{via crucis} rumo à Roda dos Enjeitados; o pai
agasalhava o filho e lhe cobria o rosto para preservá-lo do sereno.
Súbito, ``na direção do Largo da Ajuda, {[}Candinho{]} viu do lado
oposto um vulto de mulher; era a mulata fugida''. Tomado por enorme
comoção, o pai pede a um farmacêutico que cuide do filho por um instante
e dispara rumo à captura da escrava Arminda, cuja condição reificada
desponta desde o próprio nome, já que, ``possivelmente, \emph{Arminda} é
a forma feminina de \emph{Armend}, nome masculino albanês que significa
`mente dourada', a partir dos elementos \emph{ar}, que quer dizer `ouro,
dourado', e \emph{mend}, que quer dizer `mente'''\footnote{\emph{Behind
  the Name:} \emph{Arminda:
  https://www.behindthename.com/name/armend/submitted}.
  Consulta feita no dia 23/04/17.}. Como a escrava Arminda não pertence
a si mesma, sua \emph{mente} a remete primeiramente ao arbítrio de seu
senhor; depois, durante os momentos contingentes de fuga e liberdade
condicional, a \emph{mente} de Arminda é alienada em função da legião de
captores de escravos, dentre os quais desponta nosso caipora Candinho. E
Arminda, é claro, vale \emph{ouro} -- mais precisamente, cem mil-réis,
quantia polpuda que, ao menos por ora, revogaria o aborto do filho pelo
pai. Assim, no Largo da \emph{Ajuda}, o darwinismo social daquele Brasil
grotesco e cruel sentencia que a sobrevida do filho de Candinho
pressupõe os aguilhões contra os pulsos e tornozelos de Arminda -- isso
para não mencionarmos a suma tortura com a ``máscara de
folha-de-flandres, (\ldots{}) [que] tinha só três buracos, dois para ver,
um para respirar, e era fechada atrás da cabeça por um cadeado''.

Se terminasse com a captura de Arminda e a sobrevida (momentânea) do
filho do caipora Candinho, o conto já seria grotesco. Ocorre que tanto o
niilismo quanto a crítica e a apologia sociais de Machado de Assis
delegam à crueldade o papel de revelar por que o conto se intitula ``Pai
contra mãe''. Assim, enquanto se contorcia para tentar escapar das mãos
robustas de Candinho, Arminda lhe implorou ``que a soltasse pelo amor de
Deus. `Estou grávida, meu senhor!', exclamou. `Se Vossa Senhoria tem
algum filho, peço-lhe por amor dele que me solte; eu serei tua escrava,
vou servi-lo pelo tempo que quiser'''. Pobre Arminda: é por amor de seu
próprio filho que Candinho não pode soltá-la; é por amor de seu próprio
filho que Candinho não pode permitir que Arminda tenha o ventre livre.
No Brasil oitocentista, ser senhor de escravos é condição para muito
poucos. Se Candinho mal consegue alimentar Clara e seu filho, como é que
o caipora arcaria com os custos de manutenção de Arminda? Assim, com a
frieza aguçada pelo darwinismo social da guerra de todos contra todos,
Candinho grita para Arminda: ``Você é que tem culpa. Quem lhe manda
fazer filhos e fugir depois?''

Quem lhe manda fazer filhos e abortar depois, Candinho?

A obra de Machado de Assis bem poderia responder: a ordem humana, como
legado de nossa miséria, e a ordem social, reproduzindo a miséria como
nosso legado. Duas faces da mesma moeda que compra e vende legiões de
Armindas.

Quando, após muito choro e ranger de dentes, Candinho chega com a
escrava à casa de seu dono, uma cena ainda mais dantesca é parida:

\begin{quote}
Arrastada, desesperada {[}e{]} arquejando, (...) Arminda caiu no
corredor. Ali mesmo o senhor da escrava abriu a carteira e tirou os cem
mil-réis de gratificação. Cândido Neves guardou as duas notas de
cinquenta mil-réis, enquanto o senhor novamente dizia à escrava que
entrasse. No chão, onde jazia, levada do medo e da dor, e após algum
tempo de luta, a escrava abortou.
\end{quote}

Cúmplice como Candinho -- e como os leitores/espectadores --, o narrador
machadiano sentencia que, ``entre os gemidos da mãe e os gestos de
desespero do dono, Cândido Neves viu todo esse espetáculo''.

Mas, ora, que importava a Candinho o filho morto da escrava se seu filho
pôde voltar para casa? Que importavam a Candinho as palavras duras da
tia Mônica contra a fuga da escrava e contra o aborto de Arminda?
(Segundo a liturgia do poder, a vítima, por ser vítima, já é culpada;
afinal, não é o pássaro que busca a \emph{proteção} da gaiola?) Beijando
o filho (e as duas notas de cinquenta mil-réis) entre lágrimas
verdadeiras, Cândido Neves abençoa a fuga de Arminda -- a bem dizer,
Cândido Neves abençoa a escravidão que também o aguilhoa. Afinal,
sentencia o pai embalando o filho sobrevivente, ``nem todas as crianças
vingam''.

Por ora, Candinho venceu. Por ora, seu filho sobreviveu. Mas, além de os
cem mil-réis não serem eternos, o caiporismo paupérrimo de Candinho e
Clara precisa rezar, de fato, pelo pão nosso de cada dia. O narrador de
``Pai contra mãe'' bem poderia sussurrar para Candinho: \emph{amanhã,
meu caro, há de ser outro dia, já que pau que bate no filho de Arminda
também bate no filho de Clara.} Afinal, em meio à obra dialeticamente
pessimista de Machado de Assis, nem todas as crianças vingam.

\section{Referências bibliográficas}

ASSIS, Joaquim Maria Machado de. ``Pai contra mãe''. In: ROSA,
Alexandre, VASSOLER, Flávio Ricardo e LEBENSZTAYN, Ieda (Orgs.).
\emph{``Pai contra mãe'' e outros contos.} São Paulo: Editora Hedra,
2018, pp. XX-YY.

\emph{BEHIND the Name: The Etymology and History of First Names}. [Por
detrás dos nomes: a etimologia e a história dos nomes]. Disponível em:
\emph{https://www.behindthename.com/}.
Consulta feita no dia 23/04/17.

\emph{Bíblia sagrada.} Tradução dos originais mediante a versão dos
monges de Maredsous (Bélgica) pelo Centro Bíblico Católico. São Paulo:
Editora Ave-Maria, 1994.

FERREIRA, Aurélio Buarque de Holanda. \emph{Aurélio: o dicionário da
língua portuguesa.} Curitiba: Editora Positivo, 2004.