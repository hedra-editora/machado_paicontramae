\textbf{Nota dos organizadores}

\emph{Pai contra mãe e outros contos} realiza uma compilação dos livros
de contos \emph{Várias histórias} (1896)\footnote{Fazem parte de
  \emph{Várias histórias} os contos ``A cartomante''; ``Entre santos'';
  ``Uns braços''; ``Um homem célebre''; ``A desejada das gentes''; ``A
  causa secreta''; ``Trio em lá menor''; ``Adão e Eva''; ``O
  enfermeiro''; ``O diplomático''; ``Mariana''; ``Conto de escola'';
  ``Um apólogo''; ``D. Paula''; ``Viver!''; ``O cônego ou metafísica do
  estilo''.}, \emph{Páginas recolhidas} (1899)\footnote{De \emph{Páginas
  recolhidas} fazem parte os contos ``O caso da vara''; ``O
  dicionário''; ``Um erradio''; ``Eterno!''; ``Missa do galo''; ``Ideias
  do canário''; ``Lágrimas de Xerxes''; ``Papéis velhos''.} e
\emph{Relíquias de casa velha} (1906)\footnote{Fazem parte de
  \emph{Relíquias de casa velha} os contos ``Pai contra mãe''; ``Maria
  Cora''; ``Marcha fúnebre''; ``Um capitão de voluntários''; ``Suje-se,
  gordo!''; ``Umas férias''; ``Evolução''; ``Pílades e Orestes'';
  ``Anedota do \emph{Cabriolet}''.}, de Joaquim Maria Machado de Assis
(1839-1908).

Como forma de analisar alguns temas que julgamos fundamentais em meio ao
universo machadiano -- temas que, de um modo ou de outro (e sempre com o
látego da ironia em riste), perpassam os contos deste volume --,
remetemos os leitores ao ensaio ``Os inimigos do homem serão as pessoas
de sua própria casa: crítica e apologia sociais em `Pai contra mãe''',
de nossa autoria, que desponta como estudo preliminar.

Por fim, apresentamos aos leitores os fragmentos que despontam como
advertências/prefácios dos livros de contos ora reunidos de Machado de
Assis. São eles:

\textbf{(i)} ``Advertência'' a \emph{Várias histórias: }

\begin{quote}
\emph{Mon ami, faisons toujours des contes. (...) Le temps se passe, et
le conte de la vie s'achève, sans qu'on s'en aperçoive. }

Diderot\footnote{``Meu amigo, escrevamos sempre contos. (...) O tempo
  passa, e o conto da vida chega ao fim, sem nos darmos conta disso''
  (tradução livre dos organizadores). Citação da obra \emph{Salon}
  (1767), do filósofo e escritor francês Denis Diderot (1713-1784).}

As várias histórias que formam este volume foram escolhidas entre outras
e podiam ser acrescentadas, se não conviesse limitar o livro às suas
trezentas páginas. É a quinta coleção que dou ao público. As palavras de
Diderot que vão por epígrafe no rosto desta coleção servem de desculpa
aos que acharem excessivos tantos contos. É um modo de passar o tempo.
Não pretendem sobreviver como os do filósofo. Não são feitos daquela
matéria nem daquele estilo que dão aos de Mérimée\footnote{Prosper
  Mérimée (1803-1970), escritor, dramaturgo e historiador francês.} o
caráter de obras-primas e colocam os de Poe\footnote{Edgar Allan Poe
  (1809-1849), escritor, poeta e crítico literário estadunidense.} entre
os primeiros escritos da América. O tamanho não é o que faz mal a este
gênero de histórias, é naturalmente a qualidade; mas há sempre uma
qualidade nos contos, que os torna superiores aos grandes romances, se
uns e outros são medíocres: é serem curtos.
\end{quote}

\textbf{(ii)} ``Prefácio'' de \emph{Páginas recolhidas: }

\begin{quote}
\emph{Quelque diversité d'herbes qu'il y ayt, tout s'enveloppe sous le
nom de salade. }
\end{quote}

Montaigne, Essais, liv. I, cap. XLVI\footnote{``Por maior que seja a
  diversidade das verduras, tudo se torna compreensível sob o nome de
  salada''. Citação dos ensaios do escritor e pensador francês Michel de
  Montaigne (1533-1592).}

\begin{quote}
Montaigne explica pelo seu modo dele a variedade deste livro. Não há que
repetir a mesma ideia, nem qualquer outro lhe daria a graça da expressão
que vai por epígrafe. O que importa unicamente é dizer a origem destas
páginas. Umas são contos e novelas, figuras que vi ou imaginei, ou
simples ideias que me deu na cabeça reduzir a linguagem. Saíram primeiro
nas folhas volantes do jornalismo, em data diversa, e foram escolhidas
dentre muitas, por achar que ainda agora possam interessar. Também vai
aqui ``Tu só, tu, Puro Amor'', comédia escrita para as festas
centenárias de Camões\footnote{``Tu, só tu, puro amor'' é um verso da
  obra \emph{Os Lusíadas} (1572; III, 119:1), do poeta português Luís
  Vaz de Camões (1524-1580).} e representada por essa ocasião.
Tiraram-se dela cem exemplares numerados que se distribuíram por algumas
estantes e bibliotecas. Uma análise da correspondência de Renan com sua
irmã Henriqueta\footnote{Ernest Renan (1823-1892), filólogo, crítico
  literário e historiador francês, e Henriette Renan (1811-1861), sua
  irmã.}, e um debuxo do nosso antigo Senado foram dados na
\emph{Revista Brasileira}, tão brilhantemente dirigida pelo meu ilustre
e prezado amigo José Veríssimo\footnote{José Veríssimo (1857-1916),
  editor, escritor e crítico literário brasileiro.}. Sai também um
pequeno discurso, lido quando se lançou a primeira pedra da estátua de
Alencar\footnote{José de Alencar (1829-1877), escritor brasileiro.}.
Enfim, alguns retalhos de cinco anos de crônica na \emph{Gazeta de
Notícias} que me pareceram não destoar do livro, seja porque o objeto
não passasse inteiramente, seja porque o aspecto que lhe achei ainda
agora me fale ao espírito. Tudo é pretexto para recolher folhas amigas.
\end{quote}

\textbf{(iii)} ``Advertência'' a \emph{Relíquias de casa velha: }

\begin{quote}
Uma casa tem muitas vezes as suas relíquias, lembranças de um dia ou de
outro, da tristeza que passou, da felicidade que se perdeu. Supõe que o
dono pense em as arejar e expor para teu e meu desenfado. Nem todas
serão interessantes, não raras serão aborrecidas, mas, se o dono tiver
cuidado, pode extrair uma dúzia delas que mereçam sair cá fora.
Chama-lhe à minha vida uma casa, dá o nome de relíquias aos inéditos e
impressos que aqui vão, ideias, histórias, críticas, diálogos, e verás
explicados o livro e o título. Possivelmente não terão a mesma suposta
fortuna daquela dúzia de outras nem todas valerão a pena de sair cá
fora. Depende da tua impressão, leitor amigo, como dependerá de ti a
absolvição da má escolha.
\end{quote}